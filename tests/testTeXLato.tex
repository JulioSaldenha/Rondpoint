\documentclass[12pt]{article}
\usepackage{fontspec}

\setmainfont{Lato Regular}

\begin{document}
\title{\Huge Rondpoint}
\author{Sam Bell}
\date{January 12 2022}
\maketitle

\noindent\begin{minipage}{\textwidth}
\centering

\noindent\parbox{0.33\textwidth}{\noindent \large
Aa Bb Cc Dd Ee Ff Gg Hh Ii Jj Kk Ll Mm Nn Oo Pp Qq Rr Ss Tt Uu Vv Ww Xx Yy Zz

\noindent 1~2~3~4~5~6~7~8~9~0}
\end{minipage}

\vspace{1cm}

\noindent Cryptography is the practice and study of techniques for secure communication in the presence of adversarial behaviour. Morse code is a method used in telecommunication to encode text characters as standardized sequences of two different signal durations called dots and dashes. The art of reading between the lines is as old as manipulated information.

\vspace{1cm}
\noindent GCHQ is an intelligence and security organisation responsible for providing signals intelligence and information assurance to the government and armed forces of the United Kingdom. GCHQ was originally established after the First World War as the Government Code and Cypher School (GC{\&}CS) and was known under that name until 1946. During the Second World War it was located at Bletchley Park, where it was responsible for breaking the German Enigma codes. There are two main components of the GCHQ, the Composite Signals Organisation (CSO), which is responsible for gathering information, and the National Cyber Security Centre (NCSC), which is responsible for securing the UK's own communications. The Joint Technical Language Service (JTLS) is a small department and cross-government resource responsible for mainly technical language support and translation and interpreting services across government departments. It is co-located with GCHQ for administrative purposes. 

\vspace{1cm}
\noindent Plus votre texte est long et plus la police doit être lisible. Notre petite sélection de polices serif montre que ce groupe de polices d'écriture peut également faire preuve de beaucoup de variété. Les polices serif conviennent aux textes bruts, surtout s'il s'agit de textes très longs qui ne peuvent être lus jusqu'à la fin que s'ils sont faciles à déchiffrer. Mais même pour des titres un peu plus longs, il est conseillé d'utiliser des polices de caractères particulièrement lisibles. Alors que les polices sans empattement dominent sur Internet, les polices serif sont toujours le premier choix pour les produits imprimés lorsqu'il s'agit de lisibilité. 


\end{document}